\chapter{Lab and place}
\section{The MIVIA Laboratory}

\begin{wrapfigure}[2]{l}{2cm}
	\vspace{-7mm}
	\includegraphics[width=2cm]{images_not_compressed/MIVIALogo.jpg}
	\end{wrapfigure}
 for Macchine Intelligenti per il riconoscimento di Video, Immagini e Media which means intelligent machines for video, images and media recognition. The laboratory is located in Fisciano, Campania, Italy, near Salerno and Napoli as we can see below.
 \begin{figure}[h]
 \begin{center}
	 \includegraphics[width=12cm]{images_not_compressed/geoUniversity.png}
		\caption{Location of the University}
	 \end{center}
 \end{figure}
 \par Except teaching computer sciences, doctors of the lab work on pattern recognition, classification, media analysis and many parallel subjects like autonomous drones and robot vision.
 \par The laboratory is directed by Mario Vento. He is assisted by Alessia Saggese. Pasquale Foggia, Gennaro Percannella and Pierluigi Ritrovato (who is also my tutor). They all are teachers and researchers in the lab. There are also several students and graduated that work on thesis in the laboratory. I was more in contact with Antonio Greco and Raffaele Iuliano. All students and graduated was working on different projects and startups and the laboratory itself works for and with international companies.
 \par You can find all of this information on their website at this address : \url{http://mivia.unisa.it/}
 \par Because of the opportunity of double diploma, the proximity to France, the possibility to learn a new language even if we spoke English between us. I was really optimistic about the place and the laboratory.
	 
 \section{Place and life in Italy}
 
 \par The first month of internship, I lived in Carpineto, 25 minutes to the university by foot. I was enjoying the view on the mountains in the morning and the evening, but when I wanted to move during the weekends, buses didn't pass near my flat so that I had easily hours of walk to get back home. I moved on to Salerno, in the old town. Then it was easier to move the weekends. 
 \par The main touristic place around is the Amalfitan coast, which is beautiful. There are also Napoli and Pompei, which are respectively a big harbor and internatialy knows a touristic place. Globally, it is a really nice place where you can have snow in the winter and 33 degrees during weeks in the summer. 
  \begin{figure}[h]
 \begin{center}
	 \includegraphics[width=12cm]{images_not_compressed/amalfi.jpg}
		\caption{Picture of Amalfi early in the morning from the harbour}
		\label{amalfi}
	 \end{center}
 \end{figure}

 \par In the picture~\ref{amalfi} of Amlfi there is nobody, but in July, it becomes overcrowded because of tourism. Also, I found that this part of Italy is really human. It feels like you can trust people really quickly because they are really sympathizing. That is surprising when you come from France and it requires a thousand of documents to get a flat. In Salerno, the ID card is enough. People are always happy, but a bit slow sometimes.
 I took advantage of the intership to make a lot of trekking in the amalfitan coast and around Fisciano. Visit ruins, Napoli and see a lot of churches because Christianism is really present in Italy.