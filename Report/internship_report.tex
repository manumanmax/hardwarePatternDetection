\documentclass[english,a4paper,11pt]{report}
\usepackage[utf8]{inputenc}
\usepackage[T1]{fontenc}
\PassOptionsToPackage{english}{babel}
\usepackage{graphicx}
\usepackage{fullpage}
\usepackage{eso-pic}
\usepackage{cite}
\usepackage{wrapfig}
\usepackage{url}
\usepackage[english]{babel}
\usepackage{fancyhdr}
\usepackage{hyperref}
\usepackage[lastpage,user]{zref}
\cfoot{\thepage\ of \zpageref{LastPage}}
\pagestyle{fancy}
\usepackage[headsep=1cm,headheight=61pt,footskip=3cm]{geometry}


\newcommand{\HRule}{\rule{\linewidth}{0.5mm}}
 \newcommand{\ts}{\textsuperscript}
\newcommand{\blap}[1]{\vbox to 0pt{#1\vss}}
\newcommand\AtUpperLeftCorner[3]{%
  \put(\LenToUnit{#1},\LenToUnit{\dimexpr\paperheight-#2}){\blap{#3}}%
}
\newcommand\AtUpperRightCorner[3]{%
  \put(\LenToUnit{\dimexpr\paperwidth-#1},\LenToUnit{\dimexpr\paperheight-#2}){\blap{\llap{#3}}}%
}
\newcommand\AtLowerRightCorner[3]{%
  \put(\LenToUnit{\dimexpr\paperwidth-#1},\LenToUnit{#2}){#3}%
}
 
\title{\LARGE{Internship report pattern recognition UNISA Italy}}
\author{\textsc{Breton-Belz} Emmanuel - 2\ts{nd} Year Internship}
\date{\today}
\makeatletter

\renewcommand{\headrulewidth}{1pt}
\fancyhead[C]{\@author} 
\fancyhead[L]{\leftmark}
\fancyhead[R]{\includegraphics[width=2cm]{images_not_compressed/unisaLogo.jpg}}
\fancyhead[L]{{\includegraphics[width=2cm]{images_not_compressed/ensiLogo.jpg}}}
\addto\captionsenglish{
  \renewcommand{\contentsname}
    {Table of contents}
}
\renewcommand{\footrulewidth}{1pt}
\fancyfoot[R]{\leftmark}
		 
\begin{document}
	\setcounter{tocdepth}{4}
	\begin{titlepage}
	    \AddToShipoutPicture{%
	      \AtUpperLeftCorner{1.5cm}{1cm}{\includegraphics[width=4cm]{images_not_compressed/unisaLogo.jpg}}
	      \AtUpperRightCorner{1.5cm}{1cm}{\includegraphics[width=6.5cm]{images_not_compressed/ensiLogo.jpg}}
	      \AtLowerRightCorner{8cm}{3.5cm}{\parbox{7cm}{ENSICAEN \\
	        6, boulevard Maréchal Juin 
	        \\CS  45 053 – F- 14050 Caen Cedex 4\\
	        Tél. +33 (0)2 31 45 27 50\\
	        Fax +33 (0)2 31 45 27 60}} 
	    }
	 
	\begin{center}
	        \vspace*{10cm}
	        \textsc{\@title}
	        \HRule
	        \vspace*{0.5cm}
	        \large{\@author} 
	 \end{center}
 
    \vspace*{5cm}
    \begin{center}
      \makebox[\textwidth]{\includegraphics[width=\paperwidth]{images_not_compressed/uneGrandeEcole.png}}
    \end{center}
	
	\end{titlepage}
	\ClearShipoutPicture
	
	\chapter*{Thanks}
\par I would like to thank Mario Vento for hosting me in the labs, giving me the opportunity to see conferences of foreign computer scientists and following me during the internship.	
\par I thank Pierluigi Ritrovato who gave me my subjects, guided me to my targets and answered my questions.
\par I Thank Hugo Descoubes too, who helped me during the first part of my internship. 
\par Finally, I want to thank Allessia who contacted me for the papers and gave me good advices on the matching system and the international relation departement of ENSICAEN which prepared all the documents for ERASMUS facilities.
	
	
	
	\tableofcontents
	\newpage
	
	\setcounter{page}{1}
	
	\chapter{Introduction}

\par During the second year of engineering school at ENSICAEN we have to realize a 3-month internship that I decided to do in the MIVIA Lab of Salerno's university in Italy. They are specialized in image synthesis, which is my major course, moreover, I am interested in a double diploma in this university which gives me the opportunity to see the place and teachers, learn the Italian and make some contacts.
\par My internship subject has been divided in two. First a study on the Motorola HC1 helmet, used, for exemple, in the military field. The aim was to see if we could compile Linux for the helmet and install it on. Then, develop an application capable of recognizing patterns in an image and store the position, the number of patterns found and the type of each pattern. So as to determine the name of a global structure which the patterns are components.
\par In this report I will present the laboratory and the place of the internship, the study of the hardware for the helmet and the software (OpenCV mainly) for the application. What exists and where my projects are situated in their environments. After that I explain what I tried to solve the problems and what are the results.
	\chapter{Lab and place}
\section{The MIVIA Laboratory}

\begin{wrapfigure}[2]{l}{2cm}
	\vspace{-7mm}
	\includegraphics[width=2cm]{images_not_compressed/MIVIALogo.jpg}
	\end{wrapfigure}
 for Macchine Intelligenti per il riconoscimento di Video, Immagini e Media which means intelligent machines for video, images and media recognition. The laboratory is located in Fisciano, Campania, Italy, near Salerno and Napoli as we can see in the figure \ref{location}.
 \begin{figure}[h]
 \begin{center}
	 \includegraphics[width=15cm]{images_not_compressed/geoUniversity.png}
		\caption{Location of the University}
		\label{location}
	 \end{center}
 \end{figure}
 \par Besides teaching computer sciences, doctors of the lab work on pattern recognition, classification, media analysis and many parallel subjects like autonomous drones and robot vision.
 \par The laboratory is directed by Mario Vento. He is assisted by Alessia Saggese. Pasquale Foggia, Gennaro Percannella and Pierluigi Ritrovato (who is also my tutor). They are all teachers and researchers in the lab. There are also several students and graduated that work on thesis in the laboratory. I was more in contact with Antonio Greco and Raffaele Iuliano. All students and graduated was working on different projects and startups and the laboratory itself works for and with international companies.
 \par You can find all of this information on their website at this address : \url{http://mivia.unisa.it/}
 \par Because of the opportunity of double diploma, the proximity to France, the possibility to learn a new language even if we spoke English between us, I was really optimistic about the place and the laboratory.

	\chapter{Evaluation of the need}
	\chapter[Studies]{Study of the existing}

	Once again this part will be splitted in 2 because the subjects are totally differents. I am going to talk first about the helmet, that required 1 month of test and studies. I will explain the procedures in the next chapter. 
	
	\section{Helmet study}	
	\subsection{Helmet itself}
	\par I began my study by acquire some knowledges about the helmet itself. It has been release in the end of 2013, build for harsh condition, its price makes it unaccessible for the public. The army and building companies sow a good opportunity in this technologie to ease the work by bringing communication into the field.
	\par The helmet is equiped with a batterie, Wi-Fi and blutooth connections, a camera and it can be wear under the work helmet. Everything is voice commanded and very responsive thanks to Motorola's work. Windows CE 6.0 is used on the last release of the helmet image. It is good for the next section to understand that the booting system and the update system of windows CE are related. That means that you can change the file system for an update but it is windows itself that validate and copy the files on the intern memory from the SD card.
	\subsection{Embedded systems}
	\par I learned a lot about embbeded system, mainly on the boards and all the materials related to it. In our case, the materials inside the helmet are not know and not published on the Internet. Pierluigi contacted the company that brought us the helmet but they couldn't tell us which board was used in the helmet. I must have guessed which TI technology it was because the datasheet reference a TI OMAP 3 microprocessor. 
	\par That is mainly why I put my effort on the "ISEE – IGEP COM MODULE" built in with a TI OMAP 3 processor. It was at the top of the art when the helmet released and the smallest board with this processor. It corresponds well with the size of the hardware slot. In any case, if the linux kernel is compatible with the processor a cross compiled file system should boot and at least it should show an image on the screen.
	
	\subsection{Cross-compilation}
	\subsubsection{Description}
	\par The cross-compilation is a compilation for a different achitecture than the architecture that makes the computations.
	\section{Pattern recognition study}
	\subsection{OpenCV}
	\subsection{Existing code}

	\chapter{Achievements}
\par I tried to apply agile method during the developement of the application. I spitted each part in daily tasks and I reported it if they were not done. When I spent more than three days on a single task I informed Pierluigi and I looked for help in the lab. It worked really well and I can tell know that even on a project where I was single. This method matched with my rythm and my progression.
\par In the figure \ref{gantt} we can see all the progression of the internship (all the tasks are explained in the following pages). I set two important meetings for the project, but there were meetings with Mario Vento for global satisfaction and progression feedbacks. Globally I think my time was well organised and I had a regular progression.
	\begin{figure}[h]
		\begin{center}
			\includegraphics[width=15cm]{images_not_compressed/gantt.png}
			\caption{Gantt diagram}
			\label{gantt}	
		\end{center}
	\end{figure}
	\section{Installation and tests}
	

\par For keeping the idea of two distincts project, this part will a melting pot of issues and success. I will explain how I correct them if I could. I will gather the installations for OpenCV and the cross toolchain because the main idea is the same. Then I will spit the technical part in two as usual otherwise it will not be clear enough.


	\subsection{Helmet}
	\subsubsection{Cross GCC}
	
	\par It is pretty easy to install, you have to get the archieve and compile the cross compiler for your architecture. First of all I saw in the tutorial required a Ubuntu 12.10LTS\footnote{Long Lime Support} at least. I used virtual box to get a virtual machin\footnote{Software to emulate operating systems} of this distribution on my windows operating system. Next, I simply followed the tutorial that explains how to install the cross compiler. This part was really easy, after have download GNU cross G++ I ran commands and the compiler was installed.
	
	\subsubsection{Kernel compilation}
	
	\par The compilation of the kernel is almost the same as the compiler itself except that you compile for another platform. Of course the compiler handle ARM cortex A8 for TI OMAP3 otherwise. I just had to check the option for this CPU\footnote{Central processing unit} during the installation and thant the kernel I compiled was compatible with this architecture. As a result I got a file system ready to boot on an ARM device.
	
	
	\subsubsection{Tests}
	
	\par Once the file system is compiled, I had to put it inside the helmet memory. Two main problems remained, first the helmet memory was inside the helmet itself and not in the memory stick. And secondly, you can only copy files using Windows CE on the intern memory. This means that you can not erase Windows CE file system and replace it by another.
	\par That is were I found out that the project was compromised. I had few knowledges in Windows CE and the fact the helmet was not booting at first and the interface is voice commanded. I estimated the time to solve the problem longer than my internship.
	\par Thus I tryed to launch the file system on qEmu which failed. When I explained all of that to my superior they decided to change the project.
	
	\subsection{Pattern scanner}
	\subsubsection{OpenCV installation}

	\par I installed OpenCV 3.0 which is the last version and tried it. It has a base for basic image applications and modules that you can install by adding them to the module folder and compile them. Unfortunatly one module required to perform the test was not working with the 3.0 version. It is the non free module. I tryed several times but I finally decided to switch back to a previous version. After uninstall everything and install the 2.4.9 version. The module was functionnal. I tryed the code of the tutorial ~\hyperlink{opencv}{I quote earlier} and I had the first pattern recognition result. That took me 4 day in totall including the eclipse configuration that require adding all the libraries in the configuration of the compiler.
	
	\subsubsection{First build on eclipse}	
	
	\par As I just said, to make it work on eclipse I had to set the configuration for OpenCV project. Otherwise I does not compile. Right click on the project, propreties, than	go to the settings of my compiler (GCC  C++) and select includes. There I added the path to the local install folder where the OpenCV's includes are located. After that I had to set the libraries to link one by one. In the same window I went in libraries section and add their names like opencv\_core... And I added the search path of my local lib folder. Then I saved the setting and compilation was working.
	
	
	\section{Main problems}	
	\subsection{Related to the helmet}
	\par I had several issues before I began working proprely on the helmet. Some of them are related to the hardware and others to the operation system.
	\subsubsection{Hardware}
	\par The first problem we encountered was getting access to the component references of the helmet like processor, camera, mother board, etc. Pierluigi, my tutor sent mails to the company which proposed the project. Unfortunatly they was not able to gain access to this information. I looked it on the manuals and documentation that Pierluigi gave to me and the Internet. Everytime, I couldn't get the precise references, only the names of the range of the product are available.

	\par Besides, there was little issues with the helmet itself because the alimentation module was missing. That module allows us to plug the helmet directly to the outlet. To resolve the problem I had to load the batteries and plug it to the helmet to boot it.

	\subsubsection{Emulator}
	
	\par I download and install easily the emulator (qEmul) advised in the tutorial that I followed but I couldn't run the operating system on  it. The operating system started on a black screen and then nothing happend. I did not find a solution to this problem because I found out that the project was already compromised by the fact that we did not have access to the references of the components. Still I was a little bit suprised that the emulation was not working because all the steps of the tutorial was working except this one.
	
	\subsubsection{The operating system}
	
	\subsection{Related to the software}
	
	\subsubsection{The matcher}
	
	\par After 3 days of stagnation making some tests and learing about the matching system. Allessia, the thesis companion of Mario Vento made me realise that was on the wrong way using the match function of OpenCV. In fact I couldn't find multiple pattern with that method because it selects the nearest descriptors\footnote{Those which have a minimum hamming distance likeness}. The result was that I had to launch serval time the algorithm to get new matches again and again. I found the knnmatcher that perform a full matching and keeps the N best matches. That was not enough so I switched to the radius matcher that need a threshold parameter. That it finally how I introduce the threshold in the program. This value depends on the object, the distance and the conditions of the scene. In an other way the matcher keeps the matches which differ between the scene and the pattern less than the threshold.
	\par The algorithm return a vector of vector of DMatches\footnote{DMatch : A structure representing each match} so I put it in a simple vector of DMatches and it gave me very good results. The algorithm was able to find multiple time the same match on the scene. In the figure \ref{matches} we can see the fact that every time the pattern is represented it has matches in it. Not just the best ones which here corresponds to those of the first on the left.
	
	
	\begin{figure}[h]
		\begin{center}
			\includegraphics[width=10cm]{images_not_compressed/matches.jpg}
			\caption{Simple draw of radius match with 3 times the same match surround in red}
			\label{matches}	
		\end{center}
	\end{figure}
	
	\subsubsection[Multiple pattern]{Get many time the same pattern}

	\par On this part we extract two possible methods with Pierluigi. The easy one was to draw in black the image and than make the homography another time. The hardest one was to look for the key points and descriptor inside the corners and earase them of their containers. Than do the homography again. I began by the second one that sounded more relevent. 
	
	\par To be clear I have to present the DMatch structure that is detailed in the \hyperlink{structDMatch}{annexes}. The important attributs are the train index and the query index. They represent the fact that a descriptor has been found on the two images. But only the index of the descriptor is stored.
	
	\par I first thought I could get easily the points that gave me the corners after the homography. In fact I needed to remove the points that concern the pattern that has been recognized by the homography. There was no other way but to extract all the indexes of the points inside the corners I just found and remove them from the vectors. By chance, the homography take to vectors of points describing the positions of the descriptors (and the key points by the way). 
	\par On the first version of this algorithm I was looking if the point was in the bounds represented by the 4 segments of the rectangle of the pattern found. I found out when I was rotating the image that the points was removed the wrong way because the shape of the pattern was a lozenge. A lot of points was concidered out of the bounds and other that had nothing to do with the pattern was removed.
	\par I found on the internet a method~\cite{InOut} that provides the answer of my problem in O(n). This method find the number of edges that ligne from a point cross to the infinite with a polygon. If it is odd then the point is inside the polygon otherwise it is outside. The picture~\ref{OutIn} illustrate the idea.
	\begin{figure}[h]
		\begin{center}
			\includegraphics[width=10cm]{images_not_compressed/isIn.png}
			\caption{Illustration of the algorithm that tell if a point is inside a polygon}
			\label{OutIn}	
		\end{center}
	\end{figure}
	\subsubsection{Corners}
	\par OpenCV does not implement a structure for the corners. The corners correspond to the vertices of the polygone that represents the pattern. I isolated them in the first place beaucoup that is a good proof of the fact that the algorithm found efficiantly a pattern. It also helped me to find out that the first version of the removing point algorithm was not working perfectly.
	\par I let the structure as it is because I told myself that, in the future, the application could need the corners of the pattern to compute move analysis for exemple. 	
	
	\subsubsection{Others}
	
	\par My biggest parallel achivement is an algorithm that helps the user of the application to find the best threshold for his patterns. It depends of the number of pattern you want to recognize because often the threshold must be higher for certain conditions and more there is pattern in the image, more there is different conditions. 
	\par The algorithm take a scene and a pattern in input plus the number of time the pattern is represented in the image. The result of the algorithm is the first threshold that got all the patterns in the image. It computes a range of 150 values, that is why I can take 2 or 3 secondes sometimes.
	
	\begin{figure}[h]
		\begin{center}
			\includegraphics[width=10cm]{images_not_compressed/tester.png}
			\caption{Screenshot of the algorithm finding four times the MI unit pattern}
		\end{center}
	\end{figure}	
	
	\par We can see here that for units has been found with the positions of there corners
	
	\par I also had to create a video for Pierluigi to show case the possibilities of my application. Therefore I used VLC media player to stream my screen during a demonstration. I turned the stream into mp4 and used Kdenliv to make some modification and add titles. When the result was good enough I used GNOME Subtitles to write an .str file that corresponded to the video. Then I uploaded the video on youtube to share it with Pierluigi. I made some changes in the begining of the video several times. That took me a long time because I had to change manually all the subtitles that passes after the adding or the suppression.
	\par the video is accessible here : \url{https://youtu.be/q23RjtwAqDU} you can see in the figure ~\ref{video} the page were I edited the proprety and added the subtitles to make it available on youtube.
\begin{figure}[h]
		\begin{center}
			\includegraphics[width=10cm]{images_not_compressed/screenShot.png}
			\caption{Youtube edition page of the video}
			\label{video}	
		\end{center}
	\end{figure}	
	\chapter{conclusion}
%\lastpageref*{LastPages}

\par To conclued I am really happy of my internship. Even if the first project went wrong, I learnd that sometimes we have to stop before we waste our time. I appriciated the location and the communication between Pierluigi and me. I felt followed and supported during the most important phases of the project. That was both motivating and rasuring.
\par I learned a lot about compilation, that will be really usefull for the next libraries I will install. OpenCV gave me a good idea of how far we can go with image analysis. It gave me ideas for future project and I proved myself that I am capable of realize a project. I also improved my english and learn a little bit of italian. 
\par Of course being in a foreign country was an exceptionnal experience. Everyday was a new opportunity to make discoveries and see how other people live. Now I know that in the middle of a unkown country, I will have the good relfexes and I can find my way.
\par In addition to all of that, I am now sure that a double diploma really is my idea of the last year of my studies. I had doubts about it, but now I am almost sure that I want to move in different countries in my future work.
\par For all this reasons I want to thank again all the persons who help me to realize this internship.

	\clearpage

\bibliography{/home/manu/workspace/Scanner/Report/biblio}{}
	\bibliographystyle{plain}


	\listoffigures
	
	\input{summary}

	
	
\end{document}